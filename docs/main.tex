\documentclass[12pt]{article}
\usepackage{amsmath}
\usepackage{graphicx}
\usepackage{hyperref}
\usepackage{tocloft}

% Add dots in the table of contents
\renewcommand{\cftsecleader}{\cftdotfill{\cftdotsep}}

\title{Title of Your Paper}
\author{Author 1 \and Author 2 \and Author 3}
\date{\today}

\begin{document}

\maketitle

% Table of Contents
\newpage
\tableofcontents
\newpage

% Abstract
\section{Abstract}
This document provides an abstract section for your paper. Summarize your research, objectives, methodology, and key findings here.

\newpage

% Objective and Motivation
\section{Introduction}
In Switzerland, electricity suppliers, also known as distribution network
operators (Verteilnetzbetreiber), are the active participants on the wholesale
electricity market. Households, on the other hand, are not part of the
liberalized section of the market and cannot choose their suppliers. Instead,
they purchase electricity from their local supplier, who holds a monopoly over
their service area.

These suppliers can produce or buy electricity on the wholesale market to meet local
demand. The markets include:

\begin{itemize}
\item Day-Ahead Market (EPEX): The main market for securing hourly electricity
delivery for the following day. In Europe, most countries have their own
day-ahead markets, but some are coupled, meaning they trade electricity across
borders automatically. Switzerland is not part of this coupling system.
\item Intraday Market (EPEX): Used for same-day trades, though it's not very liquid
in Switzerland.
\item Futures Market (EEX): For forward contracts, though liquidity for Swiss
participants is limited. Most trading is done in Germany, as this market is 
the most liquid. 
\item Balancing Energy Market (Swissgrid): Ensures grid stability through balancing
services. Suppliers can offer or purchase balancing energy to manage imbalances
between supply and demand. The market is divided into primary, secondary, and
tertiary control reserves, with different response times and costs.
\end{itemize}

Switzerland is not part of the European Market Coupling system, meaning
suppliers must buy cross-border capacity from neighboring countries through the
Joint Allocation Office (JAO). This capacity, essential for cross-border trade,
is auctioned at different intervals—yearly, monthly, and daily—and the revenue
goes to the grid operators. The electricity supplier pays for the capacity if
opting for physical delivery, adding to the cost.

In the day-ahead market, suppliers submit bids
specifying price and quantity for electricity delivery. The market-clearing
price is set by the last accepted bid, meaning all suppliers bidding below this
price are paid the clearing price, regardless of their individual bid. This
uniform-price auction encourages suppliers to bid based on production costs,
leading to competitive outcomes.

Due to limited cross-border capacity,
Swiss suppliers must bid separately for access to neighboring countries.
Importing electricity is profitable when foreign prices are lower than Swiss
prices, while exports make sense when Swiss prices are lower than foreign
prices plus any cross-border fees.

\newpage

% Methodology
\section{Methodology}
This section details the methods and processes used in your research. Include any relevant equations, diagrams, or descriptions necessary for replicating your work.

\newpage

% Results
\section{Results}
Present your research findings here. Use figures, tables, or charts if necessary to display your data and emphasize significant outcomes.

\newpage

% References
\section{References}

\bibitem{ref1} Author, A., Author, B., and Author, C. (Year). \textit{Title of the Article}. Journal Name, Volume(Issue), Pages.
\bibitem{ref2} Author, D. (Year). \textit{Title of the Book}. Publisher.

\end{document}
